\documentclass[10pt,aspectratio=1610,table]{beamer}
\usepackage{hyperref} % Create hyperlinks in your document
% Use author-title-year bibstyle, copied from:
% https://tex.stackexchange.com/questions/249762/biblatex-verbose-style-adding-year-to-the-author-title-format
\usepackage[backend=bibtex, bibstyle=authoryear, citestyle=verbose-trad1, indexing=cite, giveninits=true, citepages=omit]{biblatex}
\newbibmacro*{cite:labeldate+extradate}{%
  \iffieldundef{labelyear}
    {}
    {\printtext[parens]{%
        \printtext[bibhyperref]{%
          \printlabeldateextra}}}}

  \renewbibmacro*{cite:name}{%
    \printnames{labelname}%
    \setunit{\printdelim{nameyeardelim}}%
    \usebibmacro{cite:labeldate+extradate}%
    \setunit*{\printdelim{nametitledelim}}}

  \renewbibmacro*{cite:idem}{%
    \bibstring[\mkibid]{idem\thefield{gender}}%
    \setunit{\printdelim{nameyeardelim}}%
    \usebibmacro{cite:labeldate+extradate}%
    \setunit{\printdelim{nametitledelim}}}
% \renewcommand*{\bibfont}{\tiny}
\addbibresource{bibliography.bib}
%\usepackage{newtxmath} % Use New TX font for math
%\usepackage{stix2} % Use stix2 font for math
\usepackage[no-math]{fontspec}
\usepackage{xeCJK} % Chinese font setting
\usepackage{tcolorbox} % Coloured boxes, for LATEX examples and theorems, etc
\tcbuselibrary{skins, breakable, theorems}
\usepackage{url} % Inserting urls
\usepackage{caption} % Customising captions in floating environments
\usepackage{verbatim}
\usepackage{multicol} % Multi-columns
\usepackage{multirow} % Multi-rows
\usepackage{booktabs} % Better tables
\usepackage{tikz} % For plot
\usepackage{pgf-pie} % For plotting pie chart
\usetikzlibrary{angles, arrows.meta, quotes, calc, intersections}
\usepackage{pgfplots}
\pgfplotsset{compat=1.16}
\usepackage{soul} % For smallcaps/striking out/highlighting
\sethlcolor{cuhklightyellow}
\usepackage{listings} % For source code
\usepackage{minted} % For source code
\usepackage{sourcecodepro} % For using source code pro as the default typewriter font
\usepackage{ifplatform} % For identifying OS platforms
\usepackage[vlined, ruled, linesnumbered]{algorithm2e} % For algorithms
\usepackage{algorithmic}
\usepackage{amsmath} % For math symbols
\usepackage{amssymb}
\usepackage{stmaryrd}
\usepackage{pifont}
\usepackage{varwidth}
\usepackage{braket} % For quantum Dirac notation
\usepackage{derivative} % For derivatives
\usepackage{microtype} % For typographical perfection

% Some packages and configurations for math
\usepackage{bm}
\renewcommand{\vec}[1]{\bm{#1}}
\DeclareMathOperator*{\E}{\mathbb{E}}
\DeclareMathOperator*{\Var}{\mathrm{Var}}
\DeclareMathOperator*{\Cov}{\mathrm{Cov}}
\DeclareMathOperator*{\argmin}{\mathrm{arg\,min\;}}
\DeclareMathOperator*{\argmax}{\mathrm{arg\,max\;}}
\def\ZZ{{\mathbb Z}}
\def\NN{{\mathbb N}}
\def\RR{{\mathbb R}}
\def\CC{{\mathbb C}}
\def\QQ{{\mathbb Q}}
\def\FF{{\mathbb F}}
\def\EE{{\mathbb E}}
\newcommand{\tr}{{\rm tr}}
\newcommand{\sign}{{\rm sign}}
\usepackage{bbm}
\usepackage{physics}
\let\div\divisionsymbol
\usepackage{blkarray}
\newcommand{\1}{\mathbbm{1}}
\newcommand{\inprod}[2]{\left\langle #1, #2 \right\rangle}
% \newcommand{\set}[1]{\left\{#1\right\}}

% Set minted style and font size, used for displaying source code
\setminted{frame=lines, breaklines, xleftmargin=16pt, linenos, fontsize=\fontsize{7}{7}, style=friendly, baselinestretch=1.2}

% For inline code, use the current font size
\makeatletter
\newcommand{\currentfontsize}{\fontsize{\f@size}{\f@baselineskip}\selectfont}
\makeatother
\setmintedinline{fontsize=\currentfontsize}

% Define CUHK theme colors
\definecolor{cuhkpurple}{RGB}{117,15,109}
\definecolor{cuhkyellow}{RGB}{221,163,0}
\definecolor{cuhkblackyellow}{RGB}{153,102,0}
\definecolor{cuhklightyellow}{RGB}{244,223,176}

% Define Theorem box style
\newtcbtheorem{theo}{Theorem}%
  {enhanced, breakable,
    colback = white, colframe = cuhkpurple, colbacktitle = cuhkpurple,
    attach boxed title to top left = {yshift = -2mm, xshift = 5mm},
    boxed title style = {sharp corners},
    fonttitle = \sffamily\bfseries}{th}

% Define Chinese fonts 
\setCJKmainfont{Noto Sans CJK TC}
\setCJKsansfont{Noto Sans CJK TC}
\setCJKmonofont{Noto Sans Mono CJK TC}

% Define English fonts 
\usefonttheme{professionalfonts}
\setmainfont[Mapping=tex-text]{Lato}
\setsansfont{Lato}
\setmonofont{Source Code Pro}

% Template theme
\useoutertheme{infolines}
\setbeamertemplate{navigation symbols}{}
\usetheme[height=10mm]{Rochester}
\usecolortheme[named=cuhkpurple]{structure}
\setbeamercolor{block title example}{use=structure,fg=white, bg=cuhkyellow}
\setbeamercolor{block body example}{use=structure,fg=black, bg=cuhklightyellow}
\setbeamercolor{alerted text}{fg=cuhkpurple, bg=cuhklightyellow}
\setbeamercolor{frametitle}{fg=white, bg=cuhkpurple}

% Some useful definitions
\newcommand{\boxalert}[1]{{
    \usebeamercolor{alerted text}\colorbox{bg}{\alert{#1}}
}}

\newcommand{\cmark}{\ding{51}}
\newcommand{\xmark}{\ding{55}}

%% Logo
\titlegraphic{
    \includegraphics[width=4cm]{The-Chinese-University-of-Hong-Hong-logo.eps}
}

